\section{Final System Architecture and Design}

\subsection{System Architecture}
When we were planning our project, we decided upon incorporating the MVC (Model-View-Controller) architecture. We were all familiar with it and it seemed like a good option based on our project's needs. We started off strong, but it slowly became muddled as time went on in the project. Particularly, the Model and View aspects blended more than they should have within our project, simply because we were focused on getting the assignment requirements complete, and the most convenient way for us resulted in this. However, we did not completely stray from it, and still made use of our Controller classes to the best of our abilities.

\subsection{Libraries, Modules \& Packages}
\label{sec:libs/mods/packs}
We made use of a lot of helpful tools. The libraries we used included ReactJS for HTML/JavaScript, Lucide for icons, Material UI for React components, bootstrap for css, and testing-library/react. One module we used was dotenv for loading environment variables.
We utilized many frameworks, such as NextJS for ReactJS, JUnit for testing Java code, Mockito for JUnit mocks, Jest for JavaScript testing, Spring and Spring Boot for quick app building, and Jakarta for web development. Lastly, a package we used the Firebase SDK.

\subsection{Coding Standards}
\subsubsection{Comments}
General code comments were placed above the lines of code they pertained to, never on the same line. Comments were meant to be clear, concise and in plain-english.

\subsubsection{Function Headers}
For Java game logic functions, we used consistent headers (in plain-english) including:
\begin{itemize}
    \item [--] Input: what the function received as input.
    \item [--] Output: what the function produced.
    \item [--] Purpose: what the function accomplished.
\end{itemize}

\subsubsection{Variables}
Following the same standard across all of our code, all variables were camel case and singular. 

\subsubsection{Database}
\label{sec:database}
\noindent Our initial plan was to use the relational SQL database management system, MySQL. However, during Sprint 1, we ran into issues with sharing a local instance of the database, so we decided to migrate to using a cloud-based NoSQL database called Firestore from Google Firebase. This was relatively easily implemented into our project.\\

\noindent With the original MySQL database, we planned for our table names to be Pascal case and singular. The primary keys were camel case followed by `ID'. The foreign keys were also camel case with a descriptor followed by `FK'.\\

\noindent In our ERD, the different tables were represented as rectangles and the fields were represented as ovals. In addition, primary keys were bolded and underlined, along with having an @ symbol in the front. Foreign keys were marked by a \# in front. Lastly, composite keys were marked like primary keys, but with an additional \# after the @ symbol. Our ERD for our original database can be found in figure \ref{fig: original ERD}. \\

\begin{figure}[hbt!]
    \centering
    \includegraphics[width=1.0\linewidth]{figures/SE Database ERD.pdf}
    \caption{ERD for our original database plan.}
    \label{fig: original ERD}
\end{figure} 

\noindent Since we switched from SQL to NoSQL, we also changed the naming conventions. We went from having tables to collections, and our collections were generally camelcase and plural. The documents within a collection had auto-generated identifiers, and the fields within documents were camelcase.\\

\noindent The model for our NoSQL database appeared similar to our original, but there were noticeable changes. Collections were made as rectangles and fields as ovals. Even though there are not technically ``relationships" in NoSQL databases, we included lines between collections to show when the equivalent to a foreign key was propagated from one collection to another. Subcollections were marked in purple. To prevent clutter and confusion, we left out the individual fields of these subcollections, as their names provided a satisfactory description of what they contained. The updated diagram for our current database can be found in figure \ref{fig:NoSQL_Model}.

\begin{figure}[hbt!]
    \centering
    \includegraphics[width=1.0\linewidth]{figures/NoSQL_Database_Model.png}
    \caption{Updated NoSQL database model.}
    \label{fig:NoSQL_Model}
\end{figure}


\pagebreak

\subsection{Technologies Used}

\subsubsection{Languages}
\noindent As stated earlier, we initially used SQL for our database, but because of the issues mentioned in the \hyperref[sec:database]{Database} section, we ended up using NoSQL. \\


\noindent For the language to run our game logic, we chose Java as we were familiar with it, and it seemed a logical choice as an object-oriented programming language. For our web UI we used Javascript and HTML, which was supplemented by ReactJS as mentioned in the \hyperref[sec:libs/mods/packs]{Libraries, Modules, and Packages} section. \\

\noindent Our initial UML class diagram that outlined our plan of the major classes in our game logic is shown in figure \ref{fig:UML}, and the UML that represented our current code is in figure \ref{fig:UML-updated}. As you can see, we had planned on having more classes than we ended up with. The classes that were dropped between the initial plan and the current code include Message, Chat, and Table. We realized that they would be unnecessary as Java classes because of how chats and table information needed to be stored in the database. All the attributes listed in these classes in the initial diagram needed to be persisted in the database anyway. \\ 

\begin{figure}[hbt!]
    \centering
    \includegraphics[width=0.8\linewidth]{figures/UML Diagram Whiteboard.pdf}
    \caption{Our first UML diagram.}
    \label{fig:UML}
\end{figure}

\begin{figure}[hbt!]
    \centering
    \includegraphics[width=0.8\linewidth]{figures/Sprint_3_UML_Diagram.png}
    \caption{Our current UML diagram.}
    \label{fig:UML-updated}
\end{figure}

\pagebreak

\subsection{UI Mockups}
Following are some of the initial UI designs for our project.

\begin{figure}[H]
    \centering
    \includegraphics[width=0.5\textwidth]{figures/Blackjack.png} \\
    \caption{This is the UI mockup for the logo of the app (illus. Emma Heiser)}
    \label{fig:logo}
\end{figure}

\begin{figure}[H]
    \centering
    \includegraphics[width=0.75\linewidth]{figures/login.png}
    \caption{This is a UI mockup of the login screen (illus. Emma Heiser)}
    \label{fig:login}
\end{figure}

\begin{figure}[H]
    \centering
    \includegraphics[width=0.75\linewidth]{figures/at-table.png}
    \caption{This is the UI mockup of creating a table (illus. Emma Heiser)}
    \label{fig:table}
\end{figure}

\begin{figure}[H]
    \centering
    \includegraphics[width=0.75\linewidth]{figures/lobby.png}
    \caption{This is the UI mockup for the lobby where players can see available tables (illus. Emma Heiser)}
    \label{fig:lobby}
\end{figure}

\begin{figure}[H]
    \centering
    \includegraphics[width=0.75\linewidth]{figures/in-game.png}
    \caption{This is the UI mockup of the game being played at a table (illus. Emma Heiser)}
    \label{fig:game}
\end{figure}

\pagebreak
