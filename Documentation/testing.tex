\section{Testing}

For our testing we will be using JUnit. We will be focused on writing good tests rather than achieving a set percentage required to commit. We believe that this is a reliable way to ensure that good software is being developed. However, if we \textit{had} to give a numerical value, we decided that 80\% coverage would be sufficient to commit.

using table \ref{tab: sprint 1 testing} , table \ref{tab: sprint 2 testing} , and table \ref{tab: sprint 3 testing} the average branch coverage for the whole project after each sprint was:

\begin{table}[!hbt]
\centering
\caption{Average branch coverage per sprint}
\label{tab: average coverage}
\begin{tabular}{|c|c|}
\hline
Sprint \#                                               & Branch Coverage (\%) \\ \hline
1                                                       & 87.65                \\ \hline
2                                                       & 81.95                \\ \hline
3                                                       & 82.22                \\ \hline
\begin{tabular}[c]{@{}c@{}}Total\\ Average\end{tabular} & 83.94                \\ \hline
\end{tabular}
\end{table}

\noindent Table 11 shows the average branch coverage for each sprint which meets the quality we aimed for. All of the statement coverage can be found in the testing section for each sprint.
