\section{Reflections on the Project}
\subsection{Chris}
I think that the most valuable thing I took away from this project was the experience of working with a team over a large project over a long period of time. If I had to go back and do it over again, I don't think I would change anything in terms of how we as a team handled the project. I tried my best to keep the team focused and on track and everyone else was more than happy to put their best foot forward during all 3 sprints. In terms of technical things there are some low level design decisions that I think I would change such as starting with firebase and skipping an incremental build of the game logic. My biggest technical regret is not properly maintaining the MVC architecture we had set to implement. In my opinion I think that this was a main cause of technical problems that cost us time and productivity. My main take away from the software engineering process is that organization and traceability are extremely important. We used some tools and practices to help with that such as kanban board and github features. Without organization or traceability it's very easy to get lost in a large project and start to produce problems or bad habits.


\subsection{Jenna}
If we had started the project today with our current knowledge, we would likely have used Firebase from the start, as everyone having access from the cloud was very convenient, and prevented having to figure out how to give everyone access to someone's local MySQL instance. It also would have been good to make a better attempt at sticking to our proposed architecture, so that we could have cleaner and more readable code.
As for our team organization, it was just about as organized as it could have been. Using the Kanban board on GitHub was a big contributor to us keeping organized, but we also just happened to be a stacked team of members that worked well together. Our communication was great, we met as much as we needed to, and we each tried to make sure everyone was doing alright with the work they were meant to do at a given time.

\subsection{Callie}
I gained valuable knowledge working with a team to plan, develop, and test a large software project. If I could have done something differently, I would've spent less time researching UI kits and packages to make the feature look nice and focused on building a new UI, which I eventually decided to do. The hardest part about this project is the necessity to learn many new tools with no prior experience in such a short time. It was my first time using Spring Boot, Firebase, and delving deeper into React so it was a lot harder to understand where errors were originating. Despite these personal challenges, as a team, we were able to have constructive weekly meetings and were communicative about our progress throughout the semester. Overall, I learned various agile practices and gained collaborative skills that I plan to continue to utilize throughout my career.

\subsection{Emma}
Reflecting on our project, I think we learned a lot that will help us in the future with other software engineering processes. If we were to start this project again today with our current knowledge, I think we would have changed our choice of database and our software development technique from the beginning. Firebase was more convenient than using MySQL and gave us fewer errors. Using this database from the beginning would have saved us the time of trying to set up MySQL and the refactoring process of switching to Firebase. Paired programming is also something I would reconsider. While it made sure more people were engaged in more areas of development for the project, it slowed our speed down, and our group met often enough for debriefs and were open to explaining areas of code we worked on that we could gain a basic understanding of other areas of the project. \par
The team was very organized throughout the project. Because of weekly scheduled meetings, we were always on the same page with each other. This made the project run smoothly from a team perspective. We were able to initially divide work for the week, then discuss and confirm with each other what our assigned stories meant from a client and technical point of view. After a few days, we could reconvene and discuss what we completed, what was causing us issues that prevented progress, and how we could solve that. \par
My main takeaways from the software engineering process are: the importance of planning and organizing user stories and the overall idea of the project before coding, and then maintenance of the code. Being able to reference our user stories and knowing the higher-level idea of what it entails helped keep me on track because the code got much larger than I expected as the sprints went on. This is also where I saw maintenance being useful to prevent the code from becoming larger with duplicate code, keep naming conventions the same to better trace components across multiple files, and organizing functions as more get added. \par


